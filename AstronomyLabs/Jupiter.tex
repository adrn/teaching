\documentclass[12pt]{article}


\usepackage{amssymb,amsmath,amsthm} 

%geometry (sets margin) and other useful packages
\usepackage[margin=1.25in]{geometry}
\usepackage{graphicx,ctable,booktabs}
\usepackage{booktabs}
\usepackage{cancel}
\usepackage{hyperref}

% Define my smaller twiddle
\newcommand{\apwsim}{\raisebox{0.2ex}{\scriptsize$\sim$\normalsize}} 

%Redefining subsections as problems
\makeatletter
\newenvironment{problem}{\@startsection
       {subsection}
       {1}
       {-.2em}
       {-3.5ex plus -1ex minus -.2ex}
       {2.3ex plus .2ex}
       {\pagebreak[3]%forces pagebreak when space is small; use \eject for better results
       \normalsize\bf\noindent{Problem }
       }
       }
       {%\vspace{1ex}\begin{center} \rule{0.3\linewidth}{.3pt}\end{center}}
       %\begin{center}\large\bf \ldots\ldots\ldots\end{center}
       }
\makeatother

% Change font size of sections
\makeatletter
\renewcommand\section{\@startsection{section}{1}{\z@}%
                                  {-3.5ex \@plus -1ex \@minus -.2ex}%
                                  {2.3ex \@plus.2ex}%
                                  {\normalfont\large\bfseries}}
\makeatother

%Fancy-header package to modify header/page numbering 
\usepackage{fancyhdr}
\pagestyle{fancy}
\lhead{Price-Whelan}
\chead{} 
\rhead{\thepage} 
\lfoot{\small\scshape Astronomy Lab 1} 
\cfoot{} 
\rfoot{\footnotesize } 
\renewcommand{\headrulewidth}{.3pt} 
\renewcommand{\footrulewidth}{.3pt}
\setlength\voffset{-0.25in}
\setlength\textheight{648pt}

\newcommand{\degrees}{\ensuremath{^\circ}}
\newcommand{\arcmin}{\ensuremath{'}}
\newcommand{\arcsec}{\ensuremath{"}}
\newcommand{\hours}{\ensuremath{^\mathrm{h}}}
\newcommand{\minutes}{\ensuremath{^\mathrm{m}}}
\newcommand{\seconds}{\ensuremath{^\mathrm{s}}}
\newcommand{\Msun}{\ifmmode {M_{\odot}}\else${M_{\odot}}$\fi}
\newcommand{\Rsun}{\ifmmode {R_{\odot}}\else${R_{\odot}}$\fi}

%%%%%%%%%%%%%%%%%%%%%%%%%%%%%%%%%%%%%%%%%%%%%%%

\begin{document}

\title{Jupiter's Motion}
\author{Adrian Price-Whelan}
\date{}%\small Revision: \today}

\maketitle

\section{Introduction}
\indent\indent You look up at the sky one night and see a bright object that you don't remember ever seeing before. Naturally, you call all of your friends and tell them to look up...but how do you describe where to look? You might tell them to look 25 degrees above the horizon, but then what? What if they live in California? The stars will all be in different positions! In most cases, astronomers use a fixed coordinate system known as \textit{equatorial coordinates} for describing the positions of the stars. The coordinate system is just like latitude and longitude, but projected onto the sky instead of onto the surface of the Earth; imagine being on the inside of a globe and seeing lines of latitude and longitude drawn on the outside. Luckily, we're not going to discuss the details of the coordinate system, but it is important to know that just like latitude and longitude, we use two \textit{angular} coordinates to specify positions on the sky. 

On the Earth though we rarely specify latitude and longitude in everyday conversation: we use states, cities, and streets to refer to rough positions. In the night sky, we can do the same, except we use constellations to refer to general areas of the sky. Unfortunately (and fortunately), we live in New York City -- you may have noticed that we can only see a handful of stars on a \emph{good} night, but that won't stop us from doing some astronomy! 

\section{Sky Map}
Go to this link, and print out a copy of the November sky map: \url{http://www.telescope.com/assets/pdf/starcharts/2012-11-starchart_bw.pdf}. This is just like a map of the Earth, except East and West are flipped (think about it -- they will be flipped whether you are looking down on the Earth or looking up at the Sky). You may find it helpful to hold the map up to the sky, and align the map with the directions on the map (North, East, South, and West). The middle of the map (the little + sign) is straight up -- what we call 'zenith.' This map is accurate for $\sim$8pm towards the beginning of the month, and $\sim$7pm for the end of the month. 

\section{Where is Jupiter?}
The goal of this ``lab'' will be to measure Jupiter's relative position to two nearby stars, Aldebaran and Capella, over the course of the entire month. This should require about 10 minutes of your time on any clear night. You should aim to take about 10 measurements over the course of the month. Because we are measuring the \emph{relative} position of Jupiter, we are just interested in how Jupiter moves on the sky relative to the background stars. Meaning, it doesn't matter what time you make your measurement as long as Jupiter is visible! If you're a night owl, Jupiter is quite high in the sky at around 12:30AM. Details on what I mean by `measurement' are below.

\section{Procedure}
Start by going outside in Columbia's campus on a clear night, and orient yourself with the sky. Line up the directions on the map with the true N,E,S,W. Next, try to identify one of the constellations shown. Cygnus (the swan) should be visible. Locate Jupiter on the map, and then find it in the sky -- it will be \textit{very} bright. Don't get confused by the many planes off to the East (JFK and LGA are both in that direction) -- it's even happened to me! You are going to take angular measurements of Jupiter's position relative to Aldebaran and Capella. Locate these stars on the chart, and try to find them in the sky. If you really have trouble, try to meet up with your lab partner and do it together. Now, you are going to take an angular measurement using some simple estimates. At arms length, the width of your pinky finger roughly corresponds to one degree, and the width of your closed fist is about 10 degrees (if this is confusing, read this: \url{http://www.oneminuteastronomer.com/860/measuring-sky/}). Hold your hand out at arms length, and estimate the angular distance between Jupiter and Aldebaran by using your hand as a distance measure. Do the same for Capella. Record your measurements, the time, the date, and sketch the position of Jupiter relative to the stars on the attached table. Make sure to provide uncertainties for your angle measurements, and make sure they are reasonable!

\textbf{Please try to do your first measurement this weekend}.

This is an ongoing lab, so it will not be due until December 5th, but the only work you have to do is make these measurements, and answer the discussion questions after a month of observations.

\section{Questions}
\noindent What motion do you observe?

\noindent Does Jupiter move with the stars? If not, why?

\noindent As the month progresses, does Jupiter get closer to Aldebaran or Capella?

\noindent As the month progresses, does Jupiter rise earlier or later in the night?

\pagebreak

\textbf{Note:} D(J-A) means ``Angular Distance from Jupiter to Aldebaran'' and D(J-C) means ``Angular Distance from Jupiter to Capella.''

\begin{table}[ht]
\centering
\small\addtolength{\tabcolsep}{24pt}

\begin{tabular}{| c | c | c | c |}
\hline
\textbf{Date} & \textbf{Time} & \textbf{D(J-A)} & \textbf{D(J-C)} \\[0.05in]
\hline
 &   &  &  \\[0.2in]\hline
 &   &  &  \\[0.2in]\hline
 &   &  &  \\[0.2in]\hline
 &   &  &  \\[0.2in]\hline
 &   &  &  \\[0.2in]\hline
 &   &  &  \\[0.2in]\hline
 &   &  &  \\[0.2in]\hline
 &   &  &  \\[0.2in]\hline
 &   &  &  \\[0.2in]\hline
 &   &  &  \\[0.2in]\hline
 &   &  &  \\[0.2in]\hline
 &   &  &  \\[0.2in]\hline
 &   &  &  \\[0.2in]\hline
 &   &  &  \\[0.2in]\hline
 &   &  &  \\[0.2in]\hline
 &   &  &  \\[0.2in]\hline
\end{tabular}
\end{table}


\end{document}

