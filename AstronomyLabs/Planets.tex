\documentclass[12pt]{article}


\usepackage{amssymb,amsmath,amsthm} 

%geometry (sets margin) and other useful packages
\usepackage[margin=1.25in]{geometry}
\usepackage{graphicx,ctable,booktabs}
\usepackage{booktabs}
\usepackage{cancel}
\usepackage{setspace}
\usepackage{lscape}
\usepackage{verbatim}

% Define my smaller twiddle
\newcommand{\apwsim}{\raisebox{0.2ex}{\scriptsize$\sim$\normalsize}} 

%Redefining subsections as problems
\makeatletter
\newenvironment{problem}{\@startsection
       {subsection}
       {1}
       {-.2em}
       {-3.5ex plus -1ex minus -.2ex}
       {2.3ex plus .2ex}
       {\pagebreak[3]%forces pagebreak when space is small; use \eject for better results
       \normalsize\bf\noindent{Problem }
       }
       }
       {%\vspace{1ex}\begin{center} \rule{0.3\linewidth}{.3pt}\end{center}}
       %\begin{center}\large\bf \ldots\ldots\ldots\end{center}
       }
\makeatother

% Change font size of sections
\makeatletter
\renewcommand\section{\@startsection{section}{1}{\z@}%
                                  {-3.5ex \@plus -1ex \@minus -.2ex}%
                                  {2.3ex \@plus.2ex}%
                                  {\normalfont\large\bfseries}}
\makeatother

%Fancy-header package to modify header/page numbering 
\usepackage{fancyhdr}
\pagestyle{fancy}

\newcommand{\degrees}{\ensuremath{^\circ}}
\newcommand{\arcmin}{\ensuremath{'}}
\newcommand{\arcsec}{\ensuremath{"}}
\newcommand{\hours}{\ensuremath{^\mathrm{h}}}
\newcommand{\minutes}{\ensuremath{^\mathrm{m}}}
\newcommand{\seconds}{\ensuremath{^\mathrm{s}}}

\newcommand{\s}[0]{\phantom{i}} %sets up \s command
\newcommand{\m}[0]{\phantom{abcde}} %sets up \m command
\providecommand{\e}[1]{\ensuremath{\times 10^{#1}}} %sets up \e command

%%%%%%%%%%%%%%%%%%%%%%%%%%%%%%%%%%%%%%%%%%%%%%%

\begin{document}

\title{What is a planet?}
\author{Adrian Price-Whelan}
\date{}%\small Revision: \today}

\maketitle

\thispagestyle{empty}

%Example problems
The International Astronomical Union (IAU) has tasked you with the job of creating a scientific definition for the word ``planet.'' You may recall that their own classification system recently changed, which resulted in Pluto losing its status as a planet. They have provided you with a list of measurements of objects -- some from within our solar system, some from other solar systems -- and ask you to decide upon a classification scheme of your choosing for these objects. You'll have to consider how you want to group them by looking at the provided properties. \textbf{You are not trying to reproduce the current classification system, so feel free to be creative.} 

Attached to this paper is a chart with data -- let's take a minute to go over the meaning of each column. 

\begin{itemize}
	\item Mass, density, and diameter should be straightforward.
	\item  The oblateness of an object is a measure of how squashed it is -- imagine sitting on a basketball. A small value of oblateness means the object is close to spherical, whereas larger values mean the object is more squashed.
	\item The semi-major axis is the longer of the two axes defined for an ellipse. 
	\item The eccentricity of an elliptical orbit is defined as $e = \sqrt{1 - (b/a)^2}$ where $a$ is the semi-major axis, and $b$ is the semi-minor axis.
	\item (ignore $\mu$)
	\item Number of moons should be straightforward.
	\item Composition indicates in which form most of the mass of the object is found. 
	\item Atmosphere designates the different elements found in the atmosphere of the object. ``Trace'' = little or no atmosphere.
\end{itemize}

\clearpage

\section{Classifying Objects}
\begin{problem}{ }
\indent\indent Find the average distance from the Sun (i.e. semi-major axis) 
of \textit{rock} objects, the average distance from the Sun of \textit{gas} objects, 
and the average distance from the Sun of \textit{rock and ice} objects.  
Are there any bodies who do not seem to fit in with the others in their 
composition class?  Which ones?  If so, recalculate the average distance
for that composition class again without this/these member(s).  Does the composition 
of an object correlate with its distance from a star? 
\end{problem}

\begin{problem}{ }
\indent\indent Identify the bodies that have the most
eccentric orbits --- you can set your own ``cutoff'' value. Look at the other
properties of the eccentric bodies. Is there anything they have in common?
\end{problem}

\begin{problem}{ }
\indent\indent Make a scatter plot of object mass versus distance from the Sun (Orbital
Radius). The x-axis will be log$_{10}a$ in units of AU, where $a$ is the semi-major axis, and the 
y-axis will be log$_{10}(Mass)$ in units of Earth masses ($M_E$). Do 
massive bodies tend to be farther or closer to their host star? What about low-mass bodies?  
What exceptions are there to this?
\end{problem}

\begin{problem}{ }
\indent\indent Make a scatter plot of object density versus orbital radius. 
The x-axis will again be log$_{10}a$ and the y-axis will be density. Does there seem
to be any relationship between density and distance from the host star? 
\end{problem}

\begin{problem}{ }
\indent\indent Make a scatter plot of number of moons versus mass of the object.  Let
your x-axis be log$_{10}(Mass)$ in units of Earth masses ($M_E$) and
your y-axis will be the number of moons.  How do these two quantities
tend to relate to each other?  What does this seem to suggest to you?
\end{problem}

\begin{problem}{ }
\indent\indent  Examine the oblateness values for the various bodies. Do you notice any patterns? Does oblateness correlate with any other characteristic?
\end{problem}

\begin{problem}{ }
\indent\indent Based on what have you have done (or any of the numbers on your datasheet), 
come up with your own classification
scheme for these objects.  Do you find obvious groupings in your plots, or
is there no kind of correlation at all?  You can have as many groupings as 
you want and as many objects in each group as you want.  Are there bodies 
that don't seem to fit into any of your classification groups? 
Take your time. There is no wrong answer to this!
Write down which bodies are put into each grouping, and explain your 
classification system in a few sentences.
\end{problem}

\begin{problem}{ }
\indent\indent Now that you've come up with a classification system with different groups,  
which of these groups do you think should be deemed \emph{planets}.  Should 
they all?  Should none of them?  What are your criteria for what is considered
to be a planet?  Write this down in a few sentences, and list the objects
that you would consider to be a planet.  Again, there is no wrong answer to 
this, just give it your best shot.
\end{problem}

\end{document}