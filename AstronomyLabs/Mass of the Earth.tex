\documentclass[12pt]{article}


\usepackage{amssymb,amsmath,amsthm} 

%geometry (sets margin) and other useful packages
\usepackage[margin=1.25in]{geometry}
\usepackage{graphicx,ctable,booktabs}
\usepackage{booktabs}
\usepackage{cancel}

% Define my smaller twiddle
\newcommand{\apwsim}{\raisebox{0.2ex}{\scriptsize$\sim$\normalsize}} 

%Redefining subsections as problems
\makeatletter
\newenvironment{problem}{\@startsection
       {subsection}
       {1}
       {-.2em}
       {-3.5ex plus -1ex minus -.2ex}
       {2.3ex plus .2ex}
       {\pagebreak[3]%forces pagebreak when space is small; use \eject for better results
       \normalsize\bf\noindent{Problem }
       }
       }
       {%\vspace{1ex}\begin{center} \rule{0.3\linewidth}{.3pt}\end{center}}
       %\begin{center}\large\bf \ldots\ldots\ldots\end{center}
       }
\makeatother

% Change font size of sections
\makeatletter
\renewcommand\section{\@startsection{section}{1}{\z@}%
                                  {-3.5ex \@plus -1ex \@minus -.2ex}%
                                  {2.3ex \@plus.2ex}%
                                  {\normalfont\large\bfseries}}
\makeatother

%Fancy-header package to modify header/page numbering 
\usepackage{fancyhdr}
\pagestyle{fancy}
\lhead{Price-Whelan}
\chead{} 
\rhead{\thepage} 
\lfoot{\small\scshape Earth, Moon, and Planets Lab} 
\cfoot{} 
\rfoot{\footnotesize Lab 1} 
\renewcommand{\headrulewidth}{.3pt} 
\renewcommand{\footrulewidth}{.3pt}
\setlength\voffset{-0.25in}
\setlength\textheight{648pt}

%%%%%%%%%%%%%%%%%%%%%%%%%%%%%%%%%%%%%%%%%%%%%%%

\begin{document}

\title{Orders of Magnitude and the Mass of the Earth \\ \large \textbf{Lab 1}}
\author{Adrian Price-Whelan}
%\date{\small Revision: \today}

\maketitle

\thispagestyle{empty}

%Example problems
\section{Powers of 10}
\indent\indent In everyday life we don't often encounter trillions, billions, or even millions of things. We may use these terms when discussing the national debt or populations of people but in most cases it is difficult to conceptualize such large values. Throughout the semester we will discuss quantities like the mass of the Sun, or the distance from the Sun to a certain planet, and these are very large quantities! But how large exactly? And how are we going to keep track of all those zeros? It will depend on the units we use! As you will see, units that are familiar to us like the meter or pound are quite inconvenient for talking about the Solar System, Galaxy, and Universe!

% As an aside, it used to be the case that we called big numbers 'astronomical,' because astronomers and mathematicians were the only people thinking about such large numbers! Of course, today that's not true because the national debt is big enough that if we had that much cash, we could put a $100 bill on EVERY STAR IN THE GALAXY, and *still* have enough money to buy several small countries.

\begin{problem}{ }
	\textit{Express your answers however you like, in any units, but you must estimate -- do not look at your phone or computer!}  \\ \\
	\textbf{How far away is the person directly across from you?} \hrulefill \\ \\
	\textbf{In the same units as above, how far away is the Sun?} \hrulefill \\ \\
	\textbf{How fast do cars move on the highway?} \hrulefill \\ \\
	\textbf{In the same units as above, how fast is the Earth moving?} \hrulefill 
\end{problem}

% This exercise is meant to do a few things: 1) they should start thinking about estimation, 2) they should start thinking about distance and velocities in space, 3) they should see how annoying it is to write so many zeros after the distance to the Sun. This will open the door for the next part on scientific notation.

\clearpage

The Sun is very far away -- much, much farther than any of your classmates (\textit{you'd better hope}!). It is so far away that you may have had trouble fitting all of the zeros in the distance answer space provided on the previous page! When we start to do calculations involving large values in general, we don't want to have to write all those zeros over and over again, right? One way we have overcome this problem is to use a set of \textit{prefixes} with common meanings to rescale our system of units. For example, if we're talking about the length of Manhattan, it might be annoying to express this value in plain old meters -- we'd have to say ``twenty-one thousand, five-hundred and sixty-five meters.'' To get around this we use the prefix \textit{kilo-} to mean ``one-thousand'' somethings, meaning we can express the length of Manhattan as (roughly) ``twenty-one kilo-meters.''

You may be familiar with many of these SI prefixes already, such as micro-, milli-, centi-, kilo-, mega-, giga-, etc. Again, we use them to rescale our units to make them more convenient to say or think about -- as a general rule when talking about big numbers, we try to keep the number we're talking about to be $<1000$ by using these prefixes. For example, we would say 500 meters, but instead of 50,000 meters, we would say 50 kilometers.

\textbf{Scientific notation} is another useful way for dealing with very large or very small numbers, and allows us to express them in a numeric short-hand. To start, remember that raising a number to some \textit{power} or \textit{exponent} means multiplying that number by itself some number of times (given by the value in the exponent): 
\begin{equation*}
	15^3 = 15\times15\times15 = 3375
\end{equation*}

We we use a Base-10 number system, so doing this exponentiation with the number 10 is very easy! If we have $10^3$, it just means writing the number 1 with 3 zeros after it because:
\begin{equation*}
	10^3 = 10\times10\times10 = 1000
\end{equation*}

But what do we do for very small numbers? Instead of raising to a \textit{positive} number, like 3, we exponentiate using a \textit{negative} number. Recall that some number, $N$, raised to the -1 power is equal to $1/N$. Therefore,

\begin{equation*}
	10^{-3} = (10^3)^{-1} = 1/1000 = 0.001
\end{equation*}

Expressing a number in scientific notation just means breaking down that number into a smaller number (between 0 and 10) multiplied by a power of ten. Let's take the number 3375 as an example:
\begin{equation*}
	3375 = 3.375\times1000 = 3.375 \times 10^3
\end{equation*}

That's it! Now we can write really big numbers in short hand, for example $2,700,000,000,000 \rightarrow 2.7\times 10^{12}$.

\clearpage

%
% UNITS
%
\section{Units}
When reporting a measurement or discussing physical quantities it is very important to include the unit. Most numbers we will deal with represent something, e.g. a distance or a length of time, and therefore requires a unit. Class is not 3 long - it is 3 \emph{hours} long; the Brooklyn Bridge is not 1.13 long - it is 1.13 \emph{miles} long.

\vspace{0.2in}
To do any calculations with more than one quantity, the units of the quantities need to agree.  You can't add centimeters and meters until you convert them to the same units.  To convert units, it is best to multiply the value by a fraction that is equal to 1.  For example, to convert 2.3 years to days:
\begin{equation}
2.3~\cancel{years} \left(\frac{365~\cancel{days}}{1~\cancel{year}}\right)\left(\frac{24~hours}{1~\cancel{day}}\right) = 20148~hours
\end{equation}
Notice that all of the units cancel except hours!

\vspace{0.2in}
Please explicitly write out your unit conversions both today and for \underline{all} calculations you do in this class.  Not only will it make your work a lot easier to read, but it will probably prevent you from making errors in your calculations.

\begin{problem}{ }
	\emph{To practice, convert the following quantities:}
	\begin{itemize}
	\item{58730} minutes =  \rule{2cm}{0.75pt} hours
	\item{38} km =  \rule{2cm}{0.75pt} cm
	\item{82} in =  \rule{2cm}{0.75pt} m (1 in = 2.54 cm)
	\end{itemize}
\end{problem}

%
% ORDERS OF MAGNITUDE
%
\section{Orders of Magnitude}
\indent\indent Scientists also often use \emph{orders of magnitude} to describe large values. The order of magnitude of a number is the ``power of ten'' that is closest to that value. Let's look at an example from before, the number $3375 = 3.375\times10^3$. If we only look at the power of 10, e.g. $10^3$, we would say that this number is ``of order one-thousand''. We will use orders of magnitude to talk about numbers when it isn't relevant or even practical to know the precise value. For example, if we are talking about the diameter of the Sun, we don't always need to know that it is 1,391,142.152603~km -- it may be good enough to know that it is \emph{of order one-million} kilometers; for estimation, it only matters that it is not one-thousand and not one-billion.
\clearpage
\begin{problem}{ }
\emph{Here are some relevant distance scales in physics, astronomy, and life in general. First write the number in scientific notation, then fill in the order of magnitude:}
\begin{center}
	\def\arraystretch{1.5}
	\begin{tabular}{| c | c | c | c |}
		\hline
		\textbf{Name} & \textbf{Number} & \textbf{Scientific Notation} & \textbf{Order of Magnitude} \\ \hline 
		Bohr Radius & $0.000000000053$~m &  &  \\ \hline 
		Human Hair & $\apwsim0.00009$~m &  &  \\ \hline 
		Human Body & $\apwsim1.83$~m &  &  \\ \hline 
		Width of USA & $\apwsim4,800,000$~m &  &  \\ \hline
		Earth to Sun & $150,000,000,000$~m &  &  \\ \hline
	\end{tabular}\linebreak\linebreak
	
	\noindent\textbf{Approximate:} How much bigger is the human body compared to the Bohr radius?\linebreak\linebreak\linebreak
	 \rule{15cm}{0.75pt}\linebreak
	 
\end{center}
\end{problem}

 When comparing or computing numbers, it is often useful to think of their orders of magnitude. In the case of adding or subtracting two large numbers, for example:
 \begin{equation*}
	8.284\times10^{13} + 1.163\times10^{7}
 \end{equation*}
These are both huge numbers -- almost 100 trillion and 10 million, respectively -- but note that one number is \emph{6 orders of magnitude larger}. The smaller number barely changes the big number, so in this case we could say:
 \begin{equation*}
	8.284\times10^{13} + 1.163\times10^{7} \approx 8.284\times10^{13}
 \end{equation*}
 based on their relative orders of magnitude.
 
 % Note that this is the equivalent of adding 1 to 1 million -- doesn't change it much!
 
Before performing any calculation it is often useful to start with an estimate based on orders of magnitude -- this gives you a ballpark answer that you can then compare to when you do the full computation. Let's do an example together:

% Start them off on this, e.g. remind them you need to multiply the number of seconds in a minute, minutes in an hour, etc.
\begin{problem}{ }
	\noindent\textbf{How many seconds are there in a year?}
	 \begin{itemize}
	 	\item What order of magnitude will the answer be? ($10^?$) \rule{3cm}{0.75pt}
		\item What is the correct answer? \rule{3cm}{0.75pt}
	\end{itemize}
\end{problem}
\clearpage
Order of magnitude estimation is very useful for sanity-checking answers, but is also a very powerful tool in astronomy. In what follows we will first estimate the mass of the Earth using some very simple arguments, and then compute the value using the results from an experiment.

%
% Mass of the Earth
%
\section{Estimating the mass of the Earth}
\indent\indent In this section, you are going to estimate the mass of the Earth using only the knowledge you have -- once again, you're not allowed to use your phone, books, or computer! Let's start by thinking about what quantities you may need. Discuss with your lab partner and create a list below of quantities you will need in order to calculate the mass of the Earth. You don't have to fill in all of the blanks, but make sure to provide units! \\
\begin{center}
	 \rule{5cm}{0.75pt} \hspace{0.2in} \rule{5cm}{0.75pt} \\
	 \vspace{0.2in}
	 \rule{5cm}{0.75pt} \hspace{0.2in} \rule{5cm}{0.75pt} \\
	 \vspace{0.2in}
	 \rule{5cm}{0.75pt} \hspace{0.2in} \rule{5cm}{0.75pt}
\end{center}

\noindent\textbf{From the quantities above, is there any way to combine the units so that everything cancels except for a unit of mass? (e.g. grams, pounds)}
\vspace{1.0in}

\noindent\textbf{What is your estimate for the mass of the Earth?}

\emph{Remember to include units, and to explain any assumptions you have made!}
\vspace{2.0in}

\clearpage

\section{Fermi problems}

\indent\indent Estimation problems like we did above with the mass of the Earth are sometimes referred to as Fermi problems, named after the physicist Enrico Fermi. He was known for his keen ability to make good approximate calculations of values with little or no actual data. Fermi is a well-known scientist for many reasons, but you should know that he was a professor in this very building! His work eventually led to the first ever nuclear fission experiment in the US right here in the basement of Pupin, and was later incorporated into the Manhattan Project (to build the first atomic bomb).

The most classic Fermi problem is \emph{How many piano tuners are in Chicago?}. You might solve this problem by estimating the population of Chicago, the number of people per household, the number of pianos per 100 households, how often a piano needs to be tuned, and then how many pianos a piano tuner could tune in one day. 

Either come up with your own Fermi problem and answer it (using similar reasoning to the above example), or choose one from the list below:

\begin{itemize}
	\item \textbf{How many employees does Duane Reade employ in Manhattan?}
	\item \textbf{If I robbed an armored car in NYC, how much money would I make?}
	\item \textbf{How many apartments in Manhattan have a water view?}
\end{itemize}

\emph{For extra credit, at the end of lab do one more of these problems!}

\clearpage

\section{Questionnaire}

\noindent\textbf{Major (s) / minor (s):}

\vspace{0.2in}

\noindent\textbf{Have you ever taken a college-level astronomy or physics class?}

\vspace{0.1in}
Yes / No

\vspace{0.2in}
\noindent\textbf{How interested are you in learning about the following topics?}

\vspace{0.1in}

How to use a telescope:
\begin{center}
	not interested \hspace{0.3in} somewhat \hspace{0.3in} very
\end{center}

\vspace{0.1in}

Solar system bodies, e.g. formation, motion, evolution:
\begin{center}
	not interested \hspace{0.3in} somewhat \hspace{0.3in} very
\end{center}

\vspace{0.1in}

Extrasolar planets, e.g. planets outside of our solar system:
\begin{center}
	not interested \hspace{0.3in} somewhat \hspace{0.3in} very
\end{center}

\vspace{0.1in}

Optics and how telescopes work:
\begin{center}
	not interested \hspace{0.3in} somewhat \hspace{0.3in} very
\end{center}

\vspace{0.1in}

Celestial navigation and the motion of the stars:
\begin{center}
	not interested \hspace{0.3in} somewhat \hspace{0.3in} very
\end{center}

\vspace{0.2in}
\noindent\textbf{Any other topics you'd like to explore in lab?}
\vspace{0.4in}


	

%- A few sentences about Fermi problems, e.g. piano tuners

%- Ask them to compute the mass of the Earth using only what they know
%-> Have an answer field for their estimate of the radius, density, mass

%- Have them do the 'g' experiment either with pendulum, or dropping a ball, then use radius above to compute the mass using $F=ma=GMm/r^2$

% http://www.ehow.com/how_7871833_radius-earth-using-sunset.html

\end{document}