\documentclass[12pt]{article}


\usepackage{amssymb,amsmath,amsthm} 

%geometry (sets margin) and other useful packages
\usepackage[margin=1.25in]{geometry}
\usepackage{graphicx,ctable,booktabs}
\usepackage{booktabs}
\usepackage{cancel}

% Define my smaller twiddle
\newcommand{\apwsim}{\raisebox{0.2ex}{\scriptsize$\sim$\normalsize}} 

%Redefining subsections as problems
\makeatletter
\newenvironment{problem}{\@startsection
       {subsection}
       {1}
       {-.2em}
       {-3.5ex plus -1ex minus -.2ex}
       {2.3ex plus .2ex}
       {\pagebreak[3]%forces pagebreak when space is small; use \eject for better results
       \normalsize\bf\noindent{Problem }
       }
       }
       {%\vspace{1ex}\begin{center} \rule{0.3\linewidth}{.3pt}\end{center}}
       %\begin{center}\large\bf \ldots\ldots\ldots\end{center}
       }
\makeatother

% Change font size of sections
\makeatletter
\renewcommand\section{\@startsection{section}{1}{\z@}%
                                  {-3.5ex \@plus -1ex \@minus -.2ex}%
                                  {2.3ex \@plus.2ex}%
                                  {\normalfont\large\bfseries}}
\makeatother

%Fancy-header package to modify header/page numbering 
\usepackage{fancyhdr}
\pagestyle{fancy}
\lhead{Price-Whelan}
\chead{} 
\rhead{\thepage} 
\lfoot{\small\scshape Astronomy Lab 1} 
\cfoot{} 
\rfoot{\footnotesize } 
\renewcommand{\headrulewidth}{.3pt} 
\renewcommand{\footrulewidth}{.3pt}
\setlength\voffset{-0.25in}
\setlength\textheight{648pt}

\newcommand{\degrees}{\ensuremath{^\circ}}
\newcommand{\arcmin}{\ensuremath{'}}
\newcommand{\arcsec}{\ensuremath{"}}
\newcommand{\hours}{\ensuremath{^\mathrm{h}}}
\newcommand{\minutes}{\ensuremath{^\mathrm{m}}}
\newcommand{\seconds}{\ensuremath{^\mathrm{s}}}
\newcommand{\Msun}{\ifmmode {M_{\odot}}\else${M_{\odot}}$\fi}
\newcommand{\Rsun}{\ifmmode {R_{\odot}}\else${R_{\odot}}$\fi}

%%%%%%%%%%%%%%%%%%%%%%%%%%%%%%%%%%%%%%%%%%%%%%%

\begin{document}

\title{Exoplanets}
\author{Adrian Price-Whelan}
\date{}%\small Revision: \today}

\maketitle

\paragraph{}
Exoplanets (or extrasolar planets) are planets orbiting stars other than our Sun. The first planets beyond our Sun were found in 1992, around
a \textit{compact stellar remnant} known as a pulsar. The first planet around a `normal' star was found in 1995. Today, we know of thousands
of exoplanetary systems thanks to the \textit{Kepler} satellite and numerous ground-based telescopes used to conduct follow-up observations.
There is a huge spread in the masses and sizes of these planets, but the smallest sizes and masses are set by the \textit{uncertainties} in our 
experiments. Luckily, we are continuing to push these limits down and eventually we will discover an Earth-like planet around a Sun-like star.

There are many methods applied to the search for exoplanets, but the three most heavily used are \textbf{radial velocity}, \textbf{transit}, and \textbf{microlensing}. In this lab, we will explore the first two methods and aim to understand how we know that the Galaxy is full of planets, some of which may be similar to our own!

% Show Kepler orary
% Discuss observational biases: sensitivity, timescales

\section{Detection via Radial Velocity}

\subsection{Finding the velocity}
\paragraph{}
Any wave emitted by an object moving toward or away from an observer is shifted in wavelength (and frequency): this is called the \textbf{Doppler shift}. You may already be familiar with this effect if you have ever noticed that the pitch (frequency) of an approaching fire truck or ambulance is \textit{higher} than it is after it passes. The effect is exactly the same for light! A light bulb moving very fast \textit{towards} you will appear bluer (higher frequency) than a light bulb moving \textit{away} from you. 
% show animation?

\paragraph{}
To quantify this, consider an object (light bulb, star, etc.) which is emitting light at a wavelength $\lambda_0$, and is
moving away from you with a velocity $v$ along the line of sight. You measure the observed wavelength $\lambda$. The
relationship between the \textit{true velocity}, $\lambda_0$, and the \textit{observed velocity}, $\lambda$ is:
\begin{equation}
\frac{\lambda - \lambda_0}{\lambda_0} = \frac{v}{c}
\end{equation}
where $c$ is the speed of light. This relationship tells us that \textbf{a measured shift in wavelength can by used to infer a velocity}. With your partner, answer
the following:\\
\vspace{10 pt}
\begin{problem}{ }
	If the velocity is towards you, the distance between the object and you is 
	decreasing with time, and thus the line-of-sight velocity is negative. In this case, is the wavelength
	you see larger or smaller than the emitted wavelength? If the original color emitted
	was green, and this effect were large, what new color might the object appear?
	\\ \\
	If the velocity is away from you, by similar arguments the velocity is positive. Now,
	does the wavelength larger or smaller? What color would the same green emitter appear?
\end{problem}

% Show radial velocity animation

\subsection{Finding the Mass}

For a solar mass central body ($M_*$) with a low-mass companion ($M_c << M_*$), e.g. a planet around a star, the orbital period $P$ of the planet and orbital
radius $R$ are related by Kepler's Third Law:
\begin{equation}
P^2 = R^3
\end{equation}
with $P$ measured in \textit{years} and $R$ in \textit{astronomical units} (AU).

The centripetal force required to hold a body of mass $m$ traveling
a speed $v$ on a circular path of radius $r$ is $m v^2 / r$. In the
star-planet system both objects travel in (roughly) circular orbits
around their common center of mass. The force required to hold them in
these orbits is provided by their mutual gravitational attraction,
which is that same for the star acting on the planet or the planet
acting on the star. Therefore the centripetal forces must be equal:
\begin{equation}
M_* \frac{v_*^2}{r_*} = M_p \frac{v_p^2}{r_p} = M_p \omega^2 r_p,
\end{equation}
where $\omega = 2 \pi / P$, $P$ is the period, and $r_*$ and $r_p$ are
the distances of the star and the planet from the center of mass,
respectively. Note that $r_* = R M_p/(M_* + M_p) \approx R M_p/M_*$
and $r_p = R M_*/(M_* + M_p) \approx R$, where $R$ is the distance
from the star to the planet.  So the equation can be rewritten
\begin{equation}
M_* \frac{v_*^2}{R M_p / M_*} = M_p \omega^2 R,
\end{equation}
or,
\begin{equation}
M_*^2 v_*^2 = M_p^2 \omega^2 R^2.
\end{equation}

We are now ready to solve for the mass of the planet. Taking the
square root of both sides, and rearranging a little, and also remembering
that $R = P^{2/3}$, we can write the mass of the planet as:

\begin{equation}
M_p = \frac{v_*}{2\pi} \Big(\frac{P}{1~\mathrm{AU}}\Big)^{1/3} M_*
\end{equation}

\paragraph{}
Answer the following with your partner: 
\begin{problem}{ }
	Holding the star's mass and the planet's orbital period constant, how does the planet's mass change
	if the velocity is higher (a faster orbit)? \\
	\\
	Now, holding the star's mass and the planet's velocity constant, how does the planet's mass change 
	if I lower the period of orbit (make a `year' shorter)?
\end{problem}

\subsection{51 Pegasi}
\paragraph{}
51 Pegasi, in the constellation Pegasus, was the first star found to
have a planet orbiting it. The planet cannot be seen directly, even
with the world's best telescopes. Instead is was detected indirectly
through its gravitational influence on the star around which it
orbits. The planet tugs on 51 Peg as it orbits, moving the star
slightly. Scientists can measure the motion of the star by looking for
a slight shift of the stellar spectrum by using the \textbf{Doppler Shift} discussed earlier. \\
\vspace{10 pt}
For 51 Peg (the attached Velocity vs. Time plot):
\begin{enumerate}
\item Find the period, $P$, and amplitude, $v_\mathrm{max}$, of
  the star's motion.

\item How big is the shift in wavelength, $\Delta \lambda$ that
  corresponds to $v_\mathrm{max}$ (use $\lambda_0 = 6000$ Angstroms)?
  The typical width of a stellar spectral line at $6000$ Angstroms is
  about $0.1$ Angstroms. How does this shift compare to the width of
  the line? 

\item Find the radius of the planet's orbit and its mass (figure out
  which equation to use based on the information you have and the
  information you want to calculate). Assume the orbit is edge-on, and
  the stellar mass is $1 M_\mathrm{Sun}$.

\item How does the mass of the planet compare to the mass of Jupiter?
  Note: $M_\mathrm{J} = 0.001 M_\mathrm{Sun}$.

\item Where would this planet be in the Solar System if it were
  orbiting the Sun? In other words, how does the size of its orbit
  compare to orbits of planets in the Solar System?
\end{enumerate}

\section{Detection via Transit}
Another method for detecting exoplanets is by observing a \textit{transit} of the planet in front of the star. Even though we can't resolve the star or planet, when the planet is in front of the star it causes a dip in the brightness of the star as measured over time. The concept is simple -- if you put something in between you and a light source, it gets dimmer!

\subsection{HD 209458}
HD209458 was the first star found to have a transiting planet. Astronomers had already determined that
HD209458 had a companion by using the radial velocity method, and careful
photometry revealed periodic, slight decrements to the brightness of the star over time. 

For HD209458b:
\begin{enumerate}
\item Show that the fraction of starlight blocked by the planet when
  it is in front of the star is
  $(R_\mathrm{planet}/R_\mathrm{star})^2$. Hint: if we could image
  them, both the planet and the star would look like circular disks in
  the sky, like the Moon or the Sun.
\item Use the second graph to determine the fraction of starlight from
  HD209458 that is blocked by the planet HD209458b.
\item What is the radius of the planet compared to the star
  $R_\mathrm{planet}/R_\mathrm{star}$ in this case?
\item If you assume the star has the same radius as the Sun, how does
 the radius of HD209458b compare to Jupiter? (The Sun's radius is
 $10$ times larger than Jupiter's.)
\end{enumerate}

\section{New, exciting exoplanet news}
\paragraph{}
Just this morning (October 17, 2012), a collaboration of European astronomers announced the discovery of a nearly Earth-mass planet (1.13$M_{Earth}$) orbiting Alpha Centauri, the closest star to the Sun! The planet is on an orbit of just 3.24~days! Given this information, and the fact that the star has approximately the same mass as the Sun, what is its orbital radius in AU? Do you think this planet could support life?

\section{Discussion}
\paragraph{}
If time allows towards the end, we'll have an informal discussion on the implications of recent
planet hunting results. This will hopefully tie in to our previous labs on astrobiology and planet taxonomy.


\end{document}

