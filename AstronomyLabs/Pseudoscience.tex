\documentclass[12pt]{article}


\usepackage{amssymb,amsmath,amsthm} 

%geometry (sets margin) and other useful packages
\usepackage[margin=1.25in]{geometry}
\usepackage{graphicx,ctable,booktabs}
\usepackage{booktabs}
\usepackage{cancel}

% Define my smaller twiddle
\newcommand{\apwsim}{\raisebox{0.2ex}{\scriptsize$\sim$\normalsize}} 

%Redefining subsections as problems
\makeatletter
\newenvironment{problem}{\@startsection
       {subsection}
       {1}
       {-.2em}
       {-3.5ex plus -1ex minus -.2ex}
       {2.3ex plus .2ex}
       {\pagebreak[3]%forces pagebreak when space is small; use \eject for better results
       \normalsize\bf\noindent{Problem }
       }
       }
       {%\vspace{1ex}\begin{center} \rule{0.3\linewidth}{.3pt}\end{center}}
       %\begin{center}\large\bf \ldots\ldots\ldots\end{center}
       }
\makeatother

% Change font size of sections
\makeatletter
\renewcommand\section{\@startsection{section}{1}{\z@}%
                                  {-3.5ex \@plus -1ex \@minus -.2ex}%
                                  {2.3ex \@plus.2ex}%
                                  {\normalfont\large\bfseries}}
\makeatother

%Fancy-header package to modify header/page numbering 
\usepackage{fancyhdr}
\pagestyle{fancy}
\lhead{Price-Whelan}
\chead{} 
\rhead{\thepage} 
\lfoot{\small\scshape Astronomy Lab 1} 
\cfoot{} 
\rfoot{\footnotesize } 
\renewcommand{\headrulewidth}{.3pt} 
\renewcommand{\footrulewidth}{.3pt}
\setlength\voffset{-0.25in}
\setlength\textheight{648pt}

\newcommand{\degrees}{\ensuremath{^\circ}}
\newcommand{\arcmin}{\ensuremath{'}}
\newcommand{\arcsec}{\ensuremath{"}}
\newcommand{\hours}{\ensuremath{^\mathrm{h}}}
\newcommand{\minutes}{\ensuremath{^\mathrm{m}}}
\newcommand{\seconds}{\ensuremath{^\mathrm{s}}}
\newcommand{\Msun}{\ifmmode {M_{\odot}}\else${M_{\odot}}$\fi}
\newcommand{\Rsun}{\ifmmode {R_{\odot}}\else${R_{\odot}}$\fi}

%%%%%%%%%%%%%%%%%%%%%%%%%%%%%%%%%%%%%%%%%%%%%%%

\begin{document}

\title{Pseudoscience}
\author{Adrian Price-Whelan}
\date{}%\small Revision: \today}

\maketitle

\section{Scientific Theory}
The main goal of science is to logically explain the world around us. We do this by constructing theories and then test these theories with repeatable experiments. Theories like the Big Bang Theory, Einstein's General Relativity, and Quantum Mechanics have testable consequences and have shown to be true up to the limits of our experimental uncertainties. Are these theories ``correct''? Almost certainly not! Newtons ``laws'' were thought to be true for centuries -- they make very clear predictions and were tested again and again and shown to be true. But then it was noticed that Mercury's orbit does something funny, something that could not be explained by Newtonian mechanics. It turns out General Relativity (GR) can explain all of classical mechanics, e.g. Newton's laws, \emph{and} this phenomena! So now we adopt GR as our theory for gravitation. Well, it turns out that over extremely large distances, it looks like GR may not be exactly true, but we don't yet have a theory that can both predict this large-scale phenomenon and explain the familiar GR or Newtonian mechanics. This is how science develops! There is no absolute truth in science -- most ``laws'' break down under certain conditions -- but we do have observations, and when enough independent observations match with a prediction of a scientific theory, we feel confident that it is a good model of the physical world, but that doesn't make it ``right.'' In fact, experiments aren't even trying to prove that a theory is right -- we can never do that -- they're simply trying to show that it is not wrong!

Science is built on the fundamental idea that any theory \emph{must} be testable -- if it is not refutable, it is not science. Unfortunately, there are many people today who want to believe something \emph{so much} that even when it has been tested and shown to be false, they still try to explain it as science. This is known as \emph{pseudoscience}. Pseudoscientific ideas are often proposed as being scientific, and try to loosely incorporate scientific concepts, but are ultimately incorrect. Examples are homeopathy, creationism, ESP / telekinesis, and astrology (unfortunately the full list is quite a bit longer). For this lab, we will discuss the implications and predictions of  \textbf{astrology} and discuss the historical relation to astronomy.

\section{Astrology}
Astrology incorporates a handful of ``theories'' that assume that there is a direct relationship between astronomical phenomena and real-life events. Of course, some astronomical phenomena \emph{do} impact every day life -- solar flares can cause communication outages, meteorites can impact the Earth, etc. -- but astrology makes even vaguer claims: that merely the positions of the stars or planets can impact your life. Astrological records date back to the 3rd century B.C.E. Many different cultures have developed their own forms of astrology, but today we will focus on ``western'' astrology which tries to make predictions based on the Zodiac signs. Do you know your zodiacal sign? Do you know what they mean? We're going to watch a short video from Carl Sagan on astrology.

\begin{problem}{ }
	Why is astrology considered a pseudoscience? What are some implications of believing in astrology in terms of what we know about physics?
\end{problem}

\begin{table}[ht]
\caption{The 12 Zodiac Signs}
\centering
\begin{tabular}{ c c c }
\hline \hline
Aries & The Ram & March 21 - April 19 \\
Taurus & The Bull & April 20 - May 20 \\
Gemini & The Twins & May 21 - June 20 \\
Cancer & The Crab & June 21 - July 22 \\
Leo & The Lion & July 23 - August 22 \\
Virgo & The Virgin & August 23 - September 22 \\
Libra & The Scales & September 23 - October 23 \\
Scorpio & The Scorpion & October 24 - November 21 \\
Sagittarius & The Archer & November 22 - December 21 \\
Capricorn & The Sea-Goat & December 22 - January 19 \\
Aquarius & The Water Carrier & January 20 - February 18 \\
Pisces & The Two Fist & February 19 - March 20 \\
\hline
\end{tabular}
\end{table}

\section{Experiment}
First off, \textbf{don't discuss with your partner}. At the end of your lab, there is a sheet with 12 horoscopes taken from the \emph{New York Post}. Read all of them, and choose the one that would have best predicted your day today. In your lab notebook, write your astrological sign next to the number of the horoscope that best predicted your day. Clearly mark this spot in your notebook, for instance, you may write ``Sagittarius, 11'' and draw a box around it. When you've done this, come up to see me and I will tell you which signs correspond to the horoscopes. Then, under the box in your notebook, either write ``Correct'' or ``Wrong.'' 

\section{Statistics}
If astrology is right, we would expect everyone to have ``correct,'' but if it's wrong, does that mean everyone will get it wrong? No! Some people will, by chance, choose the right horoscope. So how will we determine the number of ``corrects'' that we would consider significant? With statistics, of course! Let's go over some basics. When we perform an experiment that only has two outcomes (e.g. heads or tails, horoscope correct or wrong) we can use the \textbf{binomial distribution} to determine the significance of our result. Let's do an example using coin flips. 

The binomial distribution, the probability of $k$ successes in $N$ trials, can be written:
\begin{equation}
	P(k|N) = \frac{N!}{k!(N-k)!} p^{k}(1-p)^{N-k}
\end{equation}
For a coin flip, the probability of a ``success'' is just $1/2$, so in that case $p=0.5$. If

\begin{problem}{ }
	What is $p$ for our experiment? That is, what is the probability of a success?
\end{problem}
\begin{problem}{ }
	What is $N$ for our experiment? (How many ``trials'' did we do)
\end{problem}
\begin{problem}{ }
	What is the range of possible $k$ values for our experiment? In your lab notebook, plot $P(k|N)$ for all values of $k$.
\end{problem}
\begin{problem}{ }
	Which value of k gives a value closest to the percentage of the class that picked correctly?
\end{problem}
\begin{problem}{ }
	What value of k would you expect to be closest if our results were just random? What value of k would you expect to be closest if Astrology was correct?
\end{problem}
\begin{problem}{ }
	Using your answers for 4.4 and 4.5, decide if horoscopes are random or if Astrology has some merit.  Explain your answer.
\end{problem}

\clearpage

\begin{enumerate}
	\item A friend or loved one is in desperate need of cheering up and you are the one who can help. Make it your business to put the smile back on their face and, in doing so, your own worries will fade away too. % Pisces
	\item You are rarely subtle in the way you say things and today you will speak your mind without fear or favor. Leave others in no doubt at all that you are not amused by what has been going on of late. % Aries
	\item You have much to look forward to over the next few weeks, so don't waste time or energy on things that no longer matter. Whatever yesterday's problems were, it's time to put them out of your mind forever. % Cancer
	\item Not everyone can live up to your exacting standards so make allowances for those whose efforts fall short of the mark. The quality you most need to develop is patience - it won't be easy but it can be done. % Taurus
	\item What someone is saying about you is designed to make you feel bad about yourself, but you don't have to play their silly game. You know what you are capable of. You are perfect just the way you are. % Virgo
	\item You need to slow down and take life at a more leisurely pace. The planets warn that mistakes will be made unless you think things through carefully over the next few days. Remember, your health always comes first. % Gemini
	\item Someone who has a lot of charm will make a big impression on you today but are they as nice as they look? If your sixth sense tells you to be on your guard you would be wise to listen to it. % Sagittarius
	\item Find ways to streamline your daily routine so you have more time for personal matters such as affairs of the heart. Most likely you are doing things for other people that they should be doing for themselves. Enough is enough. % Libra
	\item Keep telling yourself that you are supposed to make mistakes and that it's not necessarily bad when things go wrong. The important thing now is that you learn from setbacks and make sure they don't happen again. % Leo
	\item Play by the rules and don't give your enemies an opportunity to catch you out. If you demand the highest standards of other people then, of course, they have the right to demand the highest standards of you. % Scorpio
	\item Stop worrying that the sky is about to fall - it isn't - and start doing the things that will make your existence more secure and enjoyable. For starters you should be pushing ahead with a creative project. Time's running out. % Aquarius
	\item Forget about tomorrow and focus on what you are supposed to be doing right now. Take each day as it comes and tackle a single task at a time. One small task completed is better than ten big tasks left unfinished. % Capricorn
\end{enumerate}


\end{document}

