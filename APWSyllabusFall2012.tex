\documentclass[12pt]{article}


\usepackage{amssymb,amsmath,amsthm} 

%geometry (sets margin) and other useful packages
\usepackage[margin=1.25in]{geometry}
\usepackage{graphicx,ctable,booktabs}
\usepackage{booktabs}
\usepackage{cancel}
\usepackage{url}

%Fancy-header package to modify header/page numbering 
\usepackage{fancyhdr}
\pagestyle{fancy}
\lhead{Price-Whelan}
\chead{} 
\rhead{\thepage} 
\lfoot{\small\scshape Astronomy Lab -- Fall 2012} 
\cfoot{} 
\rfoot{\footnotesize } 
\renewcommand{\headrulewidth}{.3pt} 
\renewcommand{\footrulewidth}{.3pt}
\setlength\voffset{-0.25in}
\setlength\textheight{648pt}

%%%%%%%%%%%%%%%%%%%%%%%%%%%%%%%%%%%%%%%%%%%%%%%

\begin{document}

\title{\Huge Astronomy Lab -- Fall 2012}
\author{}
\date{}%\small Revision: \today}

\maketitle

\thispagestyle{empty}

\vspace{-0.5in}
\large
\noindent\textbf{Instructor:} \hspace{0.1in} Adrian Price-Whelan  -- adrn@astro.columbia.edu\\
\textbf{Office:} \hspace{0.1in} Pupin 1420\\
\textbf{Class Time:} \hspace{0.1in} Wednesday 6-9PM\\
\textbf{Class Location:} \hspace{0.1in} Pupin Hall 1332\\
\textbf{Office Hours:} \hspace{0.1in} By appointment.\\

\noindent\textbf{Course Website:} \url{http://deimos.astro.columbia.edu/fall2012}\newline

\noindent\emph{You can know the name of a bird in all the languages of the world, but when you're finished, you'll know absolutely nothing whatever about the bird... So let's look at the bird and see what it's doing -- that's what counts.} -- Richard Feynman

\normalsize

\section*{Goals}
Science is not just a collection of facts and statements -- it is a process of exploration, observation, and critical thinking. In this vein, the goal of this class will be to cultivate your critical thinking and reasoning skills with less of a focus on raw knowledge and facts. I'd much rather you learn to estimate the mass of the Earth with pencil and paper than to memorize the names of the moons of Jupiter (though that is good to know for impressing your friends!). More concretely, we will:
\begin{itemize}
	\item Build an intuition for orders of magnitude and dealing with very large and very small numbers,
	\item Understand the difference between error and uncertainty, accuracy and precision,
	\item Understand unit conversions and why we have different systems of measurement,
	\item Gain a basic knowledge of statistics
\end{itemize}

\section*{Attendance}
You must attend lab to get credit and there are no make-up labs. Missing a lab means losing all credit for that week. Three missed labs results in a failing grade.

Being on time for labs is important because we will often start with an introduction. Missing the introduction to a lab will be counted as ``late'' and you will lose 20\% of your grade for that class.

Exceptions may be made for extraordinary circumstances or religious holidays. If you know you will need to miss a lab, you must let me know in advance. The earlier you let me know, the more flexible I can be about how and when you can make up the credit. If you miss a lab due to illness or emergency, please get in touch with me by email before the next lab session.

\section*{Materials}
Please obtain a \textbf{permanently bound} (not spiral or loose-leaf) lab notebook by our second meeting. Bring a calculator, writing implements, and your lab notebook to every meeting. You should be prepared to go outside every week, even if the weather looks bad. In the later months you will want gloves, a hat, a warm coat, etc. 

\section*{Lab Notebooks}
Your lab notebooks will be a personal record of the work you do in the class, any thoughts you might have, and any things you may find interesting. Every week at the start of lab you will be required to write 1-2 paragraphs about what you learned and remember about the previous lab session. While most lab handouts will have designated spots for you to fill in answers, you will be required to show all of your work in your lab notebooks. All calculations should have units! Feel free to add in interesting articles, plots, or things you may have learned elsewhere. I will collect your notebooks every other week and use the content when grading your labs (more on this in the next section). \textbf{Start each new lab session with a new page in your notebook with the date and lab name at the top.}

\section*{Grading}
Your grade will be determined by a combination of your lab worksheets (10\%), lab notebooks (60\%), and participation (30\%). 

\end{document}