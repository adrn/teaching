\documentclass[11pt]{article}


%---------- Uncomment one of them ------------------------------
%\usepackage[includeheadfoot, top=1in, bottom=1in, hmargin=1in]{geometry}

% \usepackage[a5paper, landscape, twocolumn, twoside,
%    left=2cm, hmarginratio=2:1, includemp, marginparwidth=43pt,
%    bottom=1cm, foot=.7cm, includefoot, textheight=11cm, heightrounded,
%    columnsep=1cm, dvips,  verbose]{geometry}
%---------------------------------------------------------------
%\usepackage{fancyhdr}
\usepackage{verbatim}
\usepackage{url}
\usepackage{epsfig}
%\pagestyle{fancy}
\usepackage{setspace}
\usepackage{amsmath}
\usepackage{graphicx}
\usepackage[includeheadfoot, top=1in, bottom=1in, hmargin=1in]{geometry}

%\usepackage{amsmath, amsfonts, amssymb,epsfig,graphicx}

\usepackage{fancyhdr}
\usepackage{url}
\pagestyle{fancy}

%\doublespacing
\singlespacing
%\onehalfspacing

\lhead{Astronomy Lab}
\chead{Tuesday, 6-9pm}
\rhead{Fall 2012}
\lfoot{Adrian Price-Whelan}
\cfoot{\thepage}
\rfoot{}

\begin{document}


\begin{center}
{\Huge Astronomy Lab: Earth, Moon, and Planets}
\end{center}



\begin{flushleft}

\section*{General Info}
\bf Instructor: Adrian Price-Whelan \\
Class Time: Tuesday 6-9 PM \\
Class Location: Classroom 1332\\
Office Hours: By appointment\\
Contact Information: ximena@astro.columbia.edu - Pupin 1410\\
TAs: We will occasionally have an TA assigned to our lab \\

% The first 10-15 minutes of each lab class will be an introduction to the topic by me and a group of students. The students will be responsible for making a short (5 minute) presentation on their take on the subject?

%\vspace{0.15in}
\section*{Overview}
Welcome to lab! We will cover a variety of topics dealing with stars, galaxies and cosmology.   Some of the activities will complement what you are learning in class, and we will try to use the telescopes as often as possible.  At the end of the semester, I really hope you have come to appreciate astronomy, and in general, the beauty of science.\\
\vspace{0.08in}


\section*{Grading}
\begin{itemize}
\item Lab Notebooks: 64\% \\
\item Presentations: 16\% \\
\item Participation: 20\% \\
\end{itemize}
\vspace{0.1in}

We will have 9 labs during the course of the semester, each will count 8\% of your final grade.  I will drop the lowest lab, meaning that lab notebooks will make up 64\% of your grade.\\
\vspace{0.1in}
We will have presentations in April. I will give you one full lab period to work on this, so the presentations will count as two labs (16\%).\\ 

\vspace{0.1in}
Participation is an essential part of the class. I expect you to be be on time and prepared to engage fully in lab. In each lab, you can get 3 points counting towards your participation grade:  1 for being on time and prepared, 1 for being engaged in lab, and 1 for asking questions and participating in discussions.  \\

\vspace{0.1in}

\section*{Attendance Policy}
You can miss one lab during the semester since your lowest lab grade will be dropped.  Due to University policy, you will fail the class if you miss more than two labs.


\vspace{0.1in}




\section*{Materials}
\begin{itemize}
\item Lab notebook
\item Scientific calculator
\item Warm clothes:  We will occasionally go up to the roof to use the telescope, please dress warmly during the colder months.
\end{itemize}

\section*{Lab Notebook Guidelines}
All of your work should be neatly written on your notebook.  I will collect lab notebooks at the end of each lab to grade that day's activity.
Here are a few guidelines to keep in mind.  You will be graded on how well you follow these instructions.
\begin{itemize}
\item Always write the name of your lab partner(s), date and title at the top of each lab.
\item Always make sure you use units! 
\item Use scientific notation when appropriate. 
\item Please be organized and try to write neatly, if I can't read it, I can't grade it.
\item I recommend using pencil instead of pen, that way you can erase any mistakes you make along the way.
\item You will be making plots throughout the semester, please use plenty of space and make sure to label the axes.
\item Show all of your steps when calculating a number so I can follow your work and give you partial credit in case you make a mistake.  Make sure to box the final answer. 
\item Answer the concluding questions section individually unless I indicate otherwise.
\end{itemize}

\section*{Miscellany}
\begin{itemize}

\vspace{0.1in}
\item Lab Partners:  Science is a collaborative endeavor so you will work in groups throughout the semester but every person should write their own answers.  Please answer the discussion section individually. 
\item Astronomy in the media: If you come across interesting astronomy-related news, feel free to bring it up in class! Be prepared to talk for 5 minutes about it. This will count for extra-credit in your participation grade. 
\item Lab Snack:  Two people will be in charged of bringing a simple snack to lab every week.
\item Please contact me if you have any questions, comments or if you are having difficulty with the material. 
\end{itemize}


\end{flushleft}
\end{document}
